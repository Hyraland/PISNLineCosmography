\documentclass[modern]{aastex62}

\usepackage{acro}
\usepackage{amsmath}

% Commands
\newcommand{\citationhere}{\textcolor{blue}{CITE}}
\newcommand{\dd}{\mathrm{d}}
\newcommand{\diff}[2]{\frac{\dd #1}{\dd #2}}
\newcommand{\fixme}[1]{\textcolor{red}{#1}}

% Shorthand
\newcommand{\mdet}{m^\mathrm{detector}}
\newcommand{\MMax}{M_\mathrm{max}}
\newcommand{\MMin}{M_\mathrm{min}}
\newcommand{\monedet}{m_1^\mathrm{detector}}
\newcommand{\monesource}{m_1^\mathrm{source}}
\newcommand{\mtwodet}{m_2^\mathrm{detector}}
\newcommand{\mtwosource}{m_2^\mathrm{source}}
\newcommand{\msource}{m^\mathrm{source}}

% Quantities
\newcommand{\alphaTrue}{1}
\newcommand{\fiveYearUncert}{5 \, \kmsMpc}
\newcommand{\gammaTrue}{3}
\newcommand{\MSourceMax}{45 \, \MSun}
\newcommand{\MSourceMin}{5 \, \MSun}
\newcommand{\None}{282}
\newcommand{\Nfive}{1463}
\newcommand{\oneYearUncert}{10 \, \kmsMpc}
\newcommand{\RZeroTrue}{100 \, \perGpcyr}

% units
\newcommand{\Gpc}{\mathrm{Gpc}}
\newcommand{\kmsMpc}{\mathrm{km} \, \mathrm{s}^{-1} \, \mathrm{Mpc}^{-1}}
\newcommand{\MSun}{M_\odot}
\newcommand{\perGpcyr}{\mathrm{Gpc}^{-3} \, \mathrm{yr}^{-1}}

% Acronym defn's
\DeclareAcronym{BH}{
  short = BH,
  long = {black hole}
}
\DeclareAcronym{BBH}{
  short = BBH,
  long = {binary black hole}
}
\DeclareAcronym{BNS}{
  short = BNS,
  long = {binary neutron star}
}
\DeclareAcronym{GW}{
  short = GW,
  long = {gravitational wave}
}
\DeclareAcronym{PISN}{
  short = PISN,
  long = {pair instability supernova}
}

\begin{document}

\title{A Two Percent Measurement }

\author[0000-0003-1540-8562]{Will M. Farr}
\affiliation{Department of Physics and Astronomy, Stony Brook University, Stony Brook NY 11794, USA}
\affiliation{Center for Computational Astronomy, Flatiron Institute, 162 5th Ave., New York NY 10010, USA}
\email{will.farr@stonybrook.edu}

\author[0000-0002-1980-5293]{Maya Fishbach}
\affiliation{Department of Astronomy and Astrophysics, University of Chicago, Chicago IL 60637, USA}
\email{mfishbach@uchicago.edu}

\author[0000-0002-0175-5064]{Daniel E. Holz}
\affiliation{Enrico Fermi Institute, Department of Physics, Department of Astronomy and Astrophysics,\\and Kavli Institute for Cosmological Physics, University of Chicago, Chicago IL 60637, USA}
\email{holz@uchicago.edu}

\begin{abstract}
%
  Joint measurements of distance and redshift can be used to constrain the
  expansion history of the universe and the associated cosmological parameters
  \citationhere{}. Merging \ac{BBH} systems are standard sirens
  \citep{Schutz1986,Holz2005}---their gravitational waveform provides direct
  information about the luminosity distance to the source.  Because gravity is
  scale-free, there is a perfect degeneracy between the source mass and redshift
  in the waveform; some non-gravitational information is necessary to break the
  degenracy and determine the redshift of the source
  \citep{Schutz1986,Chernoff1993,Finn1996,Wang1997,Holz2005,Dalal2006,Taylor2012,Messenger2012}.
  Here we suggest that the \ac{PISN}
  \citep{Heger2002,Belczynski2016,Woosley2017,Spera2017} process, thought to be
  the source of the observed upper-limit on the \ac{BH} mass in merging \ac{BBH}
  systems at $\sim \MSourceMax{}$ \citep{O1O2Population}, provides such
  information and permits a measurement of the redshift-luminosity-distance
  relation using a population of \ac{BBH} inspirals.  With realistic assumptions
  about the merger rate \citep{GWTC-1}, a mass distribution incorporating a
  sharp \ac{PISN} cutoff \citep{O1O2Population}, and measurement uncertainty
  \citep{Vitale2017} for \ac{BBH} inspirals in the Advanced LIGO and Virgo
  detectors at design sensitivity \citationhere{}, we simulate five years of
  \ac{BBH} detections. We show that after one year of operation (\fixme{put date
  here}) the \ac{BBH} population can constrain $H(z)$ to $\sim 5\%$ at a pivot
  redshift $z \simeq 0.75$.  After five years the constraint improves to $\sim
  2\%$.  This measurement relies only on general relativity and the presence of
  a cutoff mass scale that is approximately fixed or calibrated across cosmic
  time ($\Delta \MMax \lesssim 1 \, M_\odot$). When combined with a
  percent-level local measurement of the Hubble constant \citep{Chen2017} and a
  sub-percent constraint on the physical matter density from CMB measurements
  \citep{Planck2016}, this permits a $\lesssim 10\%$ measurement of the dark
  energy equation of state parameter after five years of observations.
  Observations by future ``third-generation'' \ac{GW} detectors \citationhere{},
  which can see \ac{BBH} mergers throughout the universe, would permit
  sub-percent cosmographical measurements to $z \gtrsim 4$ within one month of
  observation using this method.
%
\end{abstract}

\section{Introduction}

Some collected references (for use later):

\begin{itemize}
  \item \ac{PISN} \cite{Heger2002,Belczynski2016,Woosley2017,Spera2017}.
  \item GW170817: \cite{GW170817,GW170817-H0}
  \item Accuracy of $H_0$ measurements: \cite{GW170817-H0,Chen2017}
  \item Other pure GW mass-distribution $H_0$ methods: \citet{Taylor2012,Messenger2012}.
  \item \citet{Chernoff1993} focused on merger rate with fixed masses; similar to this, but lower mass (BNS).
  \item \citet{Finn1996} focused on something similar.
  \item \citet{Wang1997} tried to measure $\Omega_\Lambda$ based on the redshift distribution of BNS systems (i.e.\ the horizon distance tells you something about cosmology if you assume that).
\end{itemize}

\section{Method}
\label{sec:method}

\fixme{Boring details about mass ratio distribution, redshift evolution, etc,
etc.}  We draw our events from a model similar to \citet{Fishbach2018}.  In our
model, \ac{BBH} mergers occur with a rate density that is given by
%
\begin{equation}
  \label{eq:model}
  \diff{N}{m_1 \dd m_2 \dd V \dd t} =
  \begin{cases}
    R_0 \frac{\left(1 - \alpha\right) m_1^{-\alpha}}{\MMax^{1-\alpha} - \MMin^{1-\alpha}} \frac{\left(1 + \beta\right) m_2^\beta}{m_1^{1+\beta} - \MMin^{1+\beta}} \left( 1 + z \right)^{\gamma} & \MMin \leq m_2 < m_1 \leq \MMax \\
    0 & \textnormal{otherwise}
  \end{cases},
\end{equation}
%
where all quantities are referred to the source frame, $V$ is the comoving
volume (see \citet{Hogg1999} for a review of distance measures in cosmology),
the parameter $R_0$ is the volumetric merger rate at $z = 0$, $\MMin$ and
$\MMax$ are parameters giving the minimum and maximum black hole (source frame)
mass, the parameter $\alpha$ is the power-law slope of the mass function for the
more massive \ac{BH} in a \ac{BBH} merger, the parameter $\beta$ is the
power-law slope of the distribution of the smaller mass conditioned on the value
of $m_1$, and the parameter $\gamma$ controls the redshift evolution of the
merger rate.  We ignore the spins of the merging \acp{BH}; our catalogs contain
only non-spinning \acp{BH}.  Our catalogs are generated with the physically
reasonable parameters $R_0 = \RZeroTrue{}$ \citep{O1-BBH}, $\alpha =
\alphaTrue{}$ \citep{Fishbach2017}, $\gamma = \gammaTrue{}$
\citep{Fishbach2018}, $\MMin = \MSourceMin{}$
\citep{Ozel2010,Farr2011,Kreidberg2012}, and $\MMax = \MSourceMax{}$
\citep{Fishbach2017}.  We use cosmological parameters from Table 4,
``TT,TE,EE+lowP+lensing+ext'' of \citet{Planck2016}, as implemented in the
\texttt{Planck15} object of the \texttt{astropy.cosmology} module from
\citet{Astropy2018}.

\Ac{GW} detectors are directly sensitive to the mass and redshift combination
%
\begin{equation}
  \label{eq:Mdet-definition}
  \mdet \equiv \msource \left( 1 + z \right),
\end{equation}
%
where $\mdet$ is a mass parameter as measured in the detector frame, $\msource$
is the corresponding mass in the source frame, and $z$ is the redshift of the
source.  The amplitude of a gravitational wave measured at the detector depends
only on $\monedet$; $\mtwodet$; the luminosity distance, $d_L$, to the source;
and angular factors describing the orientation of the source's orbit and
position on the sky relative to the \ac{GW} antenna. Comparing parameterised
\ac{GW} waveforms
\citep[e.g.][]{Taracchini2014,Bohe2017,Smith2016,Kahn2016,Chatziioannou2017} to
the data observed by \ac{GW} detectors it is possible to produce a posterior
distribution over \ac{BBH} parameters that describes our uncertainty about the
source properties \citep{Veitch2015,GW150914-PE}; in particular, one can obtain
a marginalized three-dimensional distribution describing our knowledge of
$\monedet$, $\mtwodet$, and $d_L$, ignoring all other parameters (here we follow
the standard convention of denoting the mass of the more massive component of
the binary by $m_1$ and of the less massive component by $m_2$).  Rather than
simulate this process for the mergers in our catalog, we here approximate the
posterior that would be obtained from a proper analysis using a method similar
to \citet{Fishbach2018}, but tuned to produce uncertainties in these two
parameters that are comparable to GW170814 \citep{GW170814} at comparable
signal-to-noise ratios.  Unlike \citet{Fishbach2018}, we use \ac{GW} amplitudes
as implemented in the ``IMRPhenomPv2'' waveform \citep{Kahn2016} that
incorporate the inspiral, merger, and ringdown of the source; a source is
considered detected if its (noisily-measured) amplitude exceeds a
signal-to-noise threshold of 8 in a hypothetical detector with the .

We generate catalogs of events that correspond to all detected \ac{BBH} events
from one year and five years of three-detector observing time at design
sensitivity \citep{AdvancedLIGO,AdvancedVIRGO}.  The first year's observations
of the five-year catalog are identical to the one-year catalog. The catalogs
contain \None{} and \Nfive{} events.  Figure \ref{fig:true-det-masses-dLs} shows
the true (detector frame) masses and luminosity distances for the detected
events; the events range from $\sim 5 \, \MSun$ to $80 \, \MSun$ in mass
(detector frame) and reach to $d_L \sim 6 \, \Gpc$.  Our approximation to the
posterior produces the set of inferred masses and distances shown in Figure
\ref{fig:obs-det-masses-dLs}.

\begin{figure}
  \plotone{plots/m1-dL-true}
%
  \caption{\label{fig:true-det-masses-dLs} The true (detector frame) masses and
  luminosity distances for the one year (dark) and five year (light) catalogs
  described in Section \ref{sec:method}.  The effect of redshifting on the mass
  can clearly be seen; the maximum source-frame mass is $\MSourceMax$.  The
  solid black line gives the maximum detector-frame mass as a function of
  luminosity distance for the cosmology used to generate the catalog
  \citep{Planck2016}; the dashed lines give the same relation but for a 10\%
  change in $H_0$.}
%
\end{figure}

\begin{figure}
  \plotone{plots/m1-dL-obs}
%
  \caption{\label{fig:obs-det-masses-dLs} The inferred (detector frame) masses
  and luminosity distances for the one-year catalog (dark) and five-year catalog
  (light).  The points are the posterior mean and bars give the posterior
  standard deviation with a flat prior on $\monedet$ and $d_L$.  The solid black
  line gives the maximum detector-frame mass as a function of luminosity
  distance for the cosmology used to generate the catalog.}
%
\end{figure}

\section{Conclusions}

\acknowledgments

We thank Stephen Feeney for providing a sounding board for the methods discussed
in this paper.  We acknowledge the 2018 April APS Meeting and Barley's Brewing
Company in Columbus, OH, USA where this work was originally conceived.

\bibliography{pisn-line}

\end{document}
