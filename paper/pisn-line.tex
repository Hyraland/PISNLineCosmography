\documentclass[modern]{aastex62}

\usepackage{acro}
\usepackage{amsmath}

% Commands
\newcommand{\citationhere}{\textcolor{blue}{CITE}}
\newcommand{\dd}{\mathrm{d}}
\newcommand{\diff}[2]{\frac{\dd #1}{\dd #2}}
\newcommand{\fixme}[1]{\textcolor{red}{#1}}

% Shorthand
\newcommand{\mdet}{m^\mathrm{detector}}
\newcommand{\MMax}{M_\mathrm{max}}
\newcommand{\MMin}{M_\mathrm{min}}
\newcommand{\monedet}{m_1^\mathrm{detector}}
\newcommand{\monesource}{m_1^\mathrm{source}}
\newcommand{\mtwodet}{m_2^\mathrm{detector}}
\newcommand{\mtwosource}{m_2^\mathrm{source}}
\newcommand{\msource}{m^\mathrm{source}}

% Quantities
\newcommand{\MSourceMin}{5 \, M_\odot}
\newcommand{\MSourceMax}{45 \, M_\odot}
\newcommand{\OOneOTwoAlpha}{0.4^{+1.3}_{-1.9}}
\newcommand{\OOneOTwoMergerRate}{65^{+76}_{-34} \, \perGpcyr}
\newcommand{\SigmaHPvtOneYear}{4\%}
\newcommand{\SigmaHPvtFiveYear}{1.5\%}
\newcommand{\SigmaWDEFiveYear}{8\%}
\newcommand{\zpivot}{0.75}

% units
\newcommand{\Gpc}{\mathrm{Gpc}}
\newcommand{\kmsMpc}{\mathrm{km} \, \mathrm{s}^{-1} \, \mathrm{Mpc}^{-1}}
\newcommand{\MSun}{M_\odot}
\newcommand{\perGpcyr}{\mathrm{Gpc}^{-3} \, \mathrm{yr}^{-1}}

% Acronym defn's
\DeclareAcronym{BH}{
  short = BH,
  long = {black hole}
}
\DeclareAcronym{BBH}{
  short = BBH,
  long = {binary black hole}
}
\DeclareAcronym{BNS}{
  short = BNS,
  long = {binary neutron star}
}
\DeclareAcronym{GW}{
  short = GW,
  long = {gravitational wave}
}
\DeclareAcronym{GWTC1}{
  short = GWTC1,
  long = {gravitational wave transient catalog 1}
}
\DeclareAcronym{PISN}{
  short = PISN,
  long = {pair instability supernova}
}

\begin{document}

\title{A Two Percent Measurement }

\author[0000-0003-1540-8562]{Will M. Farr}
\affiliation{Department of Physics and Astronomy, Stony Brook University, Stony Brook NY 11794, USA}
\affiliation{Center for Computational Astronomy, Flatiron Institute, 162 5th Ave., New York NY 10010, USA}
\email{will.farr@stonybrook.edu}

\author[0000-0002-1980-5293]{Maya Fishbach}
\affiliation{Department of Astronomy and Astrophysics, University of Chicago, Chicago IL 60637, USA}
\email{mfishbach@uchicago.edu}

\author[0000-0002-0175-5064]{Daniel E. Holz}
\affiliation{Enrico Fermi Institute, Department of Physics, Department of Astronomy and Astrophysics,\\and Kavli Institute for Cosmological Physics, University of Chicago, Chicago IL 60637, USA}
\email{holz@uchicago.edu}

\begin{abstract}
%
  Joint measurements of distance and redshift can be used to constrain the
  expansion history of the universe and the associated cosmological parameters
  \citationhere{}. Merging \ac{BBH} systems are standard sirens
  \citep{Schutz1986,Holz2005}---their gravitational waveform provides direct
  information about the luminosity distance to the source.  Because gravity is
  scale-free, there is a perfect degeneracy between the source mass and redshift
  in the waveform; some non-gravitational information is necessary to break the
  degenracy and determine the redshift of the source
  \citep{Schutz1986,Chernoff1993,Finn1996,Wang1997,Holz2005,Dalal2006,Taylor2012,Messenger2012,GW170817-H0}.
  Here we suggest that the \ac{PISN}
  \citep{Heger2002,Belczynski2016,Woosley2017,Spera2017} process, thought to be
  the source of the observed upper-limit on the \ac{BH} mass in merging \ac{BBH}
  systems at $\sim \MSourceMax{}$ \citep{O1O2Population}, provides such
  information and permits a measurement of the redshift-luminosity-distance
  relation using a population of \ac{BBH} inspirals.  With realistic assumptions
  about the merger rate \citep{GWTC-1}, a mass distribution incorporating a
  sharp \ac{PISN} cutoff \citep{O1O2Population}, and measurement uncertainty
  \citep{Vitale2017} for \ac{BBH} inspirals in the Advanced LIGO and Virgo
  detectors at design sensitivity \citationhere{}, we simulate five years of
  \ac{BBH} detections. We show that after one year of operation at design
  sensitivity (\fixme{put date here}) the \ac{BBH} population can constrain
  $H(z)$ to $\SigmaHPvtOneYear$ at a pivot redshift $z \simeq \zpivot$.  After
  five years (\fixme{date}) the constraint improves to $\SigmaHPvtFiveYear$.
  This measurement relies only on general relativity and the presence of a
  cutoff mass scale that is approximately fixed or calibrated across cosmic time
  ($\Delta \MMax \lesssim 1 \, M_\odot$); it is independent of any distance
  ladder or cosmological model. When combined with a percent-level local
  measurement of the Hubble constant \citep{Chen2017} and a sub-percent
  constraint on the physical matter density from CMB measurements
  \citep{Planck2016} in a $w\mathrm{CDM}$ cosmological model, the dark energy
  equation of state parameter is determined with $\SigmaWDEFiveYear$
  uncertainty. Observations by future ``third-generation'' \ac{GW} detectors
  \citationhere{}, which can see \ac{BBH} mergers throughout the universe, would
  permit sub-percent cosmographical measurements to $z \gtrsim 4$ within one
  month of observation.
%
\end{abstract}

\section*{ }

The \ac{GWTC1} contains ten binary black hole merger events observed during
Advanced LIGO and Advanced VIRGO's first and second observing runs
\citep{GWTC-1}. Modeling of this population suggests a precipitous drop in the
merger rate for primary black hole masses larger than $\sim \MSourceMax{}$
\citep{Fishbach2017,GWTC-1}.  A possible explanation for this drop is the
\ac{PISN} process \citationhere{}. This process occurs in the cores of massive
stars (helium core masses $30$--$133 \, \MSun$ \citep{Woosley2017}) when the
core temperature becomes sufficiently high to permit the production of
electron-positron pairs; pair production softens the equation of state of the
core, leading to a collapse which is halted by nuclear burning
\citep{Heger2002}.  The energy produced can either unbind the star, leaving no
\ac{BH} remnant, or drive a mass-loss pulse that reduces the mass of the star
until the \ac{PISN} is halted, leading to remnant masses $\sim \MSourceMax{}$.
Modeling suggests that the upper limit on the remnant mass may vary by less than
\fixme{XX} with redshift for $0 \leq z \lesssim 2$ \citep{Belczynski2016}.

Compact object mergers that emit gravitational waves have a universal
characteristic peak luminosity $c^5/G \simeq 3.6 \times 10^{59} \, \mathrm{erg}
\, \mathrm{s}^{-1}$ that enables direct measurements of the luminosity distance
to these sources \citep{Schutz1986}.  They are standard sirens \citep{Holz2005}.
However, the source-frame mass of the merging objects is degenerate with the
redshift; the waveform depends only on the redshifted mass in the detector
frame, $m_\mathrm{det} = m_\mathrm{source} (1 + z)$.  General relativity
predicts the gravitational waveforms of stellar-mass \ac{BBH} mergers.  Using
parameterized models of these waveforms
\citep{Taracchini2014,Kahn2016,Bohe2017,Chatziioannou2017}, it will be possible
to measure the detector-frame masses with $\sim 20\%$ uncertainty and luminosity
distances \citep{Hogg1999} with $\sim 50\%$ uncertainty for a source near the
detection threshold in Advanced LIGO and Anvanced Virgo at design sensitivity
\citep{Vitale2017} \citationhere{}.  The uncertainty in these parameters scales
inversely with the signal-to-noise ratio of a source \citep{Vitale2017}.

If the \ac{BBH} merger rate follows the star formation rate
\citep{Fishbach2018,O1O2Population}, the primary mass distribution follows a
declining power law $m_1^{-\alpha}$ with $\alpha \simeq 0.75$ for $m_1 \lesssim
\MSourceMax{}$, the mass ratio distribution is flat, and the three-detector duty
cycle is $\sim 50\%$ then Advanced LIGO and Advanced Virgo should detect $\sim
1000$ \ac{BBH} mergers per year at design sensitivity over a range of redshifts
$0 \leq z \lesssim 1.5$.  The typical detected merger will have a redshift $z
\sim 0.75$.  If we assume that $\sim 1/2$ these detections are informative about
the redshifted upper limit on the remnant mass in the detector frame,
$m_\mathrm{max,det} = m_\mathrm{max,source} \left(1 + z\left( d_L \right)
\right)$, that the combined uncertainty is $1/\sqrt{N}$ smaller than the
single-measurement uncertainty, and that most detections are near threshold,
then we are dominated by the $\sim 50\%$ distance uncertainty and can achieve an
absolute distance-redshift measurement (i.e.\ constrain the local expansion
rate, $H(z)$, for $z \simeq 0.75$) at the $50 \% / \sqrt{1000/2} \simeq 2.2 \%$
level after one year, and the $1 \%$ level after five years of \ac{BBH} merger
observations at design sensitivity.

\acknowledgments

We thank Stephen Feeney for providing a sounding board for the methods discussed
in this paper.  We acknowledge the 2018 April APS Meeting and Barley's Brewing
Company in Columbus, OH, USA where this work was originally conceived.

\bibliography{pisn-line}

\appendix

\section{Parameter Estimation}

Blah.

\end{document}
