\documentclass[modern]{aastex62}

\usepackage{acro}

% Quantities
\newcommand{\oneYearUncert}{10 \, \kmsMpc}
\newcommand{\fiveYearUncert}{5 \, \kmsMpc}

% units
\newcommand{\kmsMpc}{\mathrm{km} \, \mathrm{s}^{-1} \, \mathrm{Mpc}^{-1}}
\newcommand{\MSun}{M_\odot}

% Acronym defn's
\DeclareAcronym{BH}{
  short = BH,
  long = {black hole}
}
\DeclareAcronym{BBH}{
  short = BBH,
  long = {binary black hole}
}
\DeclareAcronym{BNS}{
  short = BNS,
  long = {binary neutron star}
}
\DeclareAcronym{GW}{
  short = GW,
  long = {gravitational wave}
}
\DeclareAcronym{PISN}{
  short = PISN,
  long = {pair instability supernova}
}

\begin{document}

\title{The Pair Instability Supernova Mass Absorption Line: Enabling a Pure
Gravitational Wave Measurement of the Hubble Constant}

\author[0000-0003-1540-8562]{Will M. Farr}
\affiliation{Department of Physics and Astronomy, Stony Brook University, Stony Brook NY 11794, USA}
\affiliation{Center for Computational Astronomy, Flatiron Institute, 162 5th Ave., New York NY 10010, USA}
\email{will.farr@stonybrook.edu}

\author[0000-0002-1980-5293]{Maya Fishbach}
\affiliation{Department of Astronomy and Astrophysics, University of Chicago, Chicago IL 60637, USA}
\email{mfishbach@uchicago.edu}

\author[0000-0002-0175-5064]{Daniel E. Holz}
\affiliation{Enrico Fermi Institute, Department of Physics, Department of Astronomy and Astrophysics,\\and Kavli Institute for Cosmological Physics, University of Chicago, Chicago IL 60637, USA}
\email{holz@uchicago.edu}

\begin{abstract}
%
  Gravitational interactions are scale-free.  Because of this freedom, there is
  a perfect degeneracy between the mass scale and redshift of a gravitational
  wave source; some non-gravitational process is necessary to provide redshift
  information. \Ac{GW} observations do provide direct information about the
  luminosity distance of a \ac{GW} source.  Here we suggest that the \ac{PISN}
  process, thought to impose a sharp upper-limit on the source-frame mass of
  \ac{BH} stellar remnants, provides an ``absorption line'' that permits a
  statistical measurement of the redshift-luminosity-distance relation using a
  population of \ac{BBH} inspirals.  Assuming that the intrinsic scatter in the
  mass upper-limit of the \ac{BH} mass distribution imposed by the \ac{PISN}
  process is $\lesssim 1\,\MSun$, we simulate one and five years of Advanced
  LIGO \ac{BBH} detections with realistic uncertainties is mass and distance
  measurements and show that analysis of these populations permits determination
  of the Hubble constant to $\pm \oneYearUncert$ after one and $\pm
  \fiveYearUncert$ after five years of observation.  These uncertainties are
  competitive with, if not better, than obtainable through any other known
  \ac{GW} observational technique with this data set, including observations of
  \ac{BNS} mergers \emph{with} electromagnetic counterparts like GW170817.
%
\end{abstract}

\section{Introduction}

\section{Method}

\section{Simulated Observations}

\section{Conclusions}

\acknowledgments

We thank Stephen Feeney for providing a sounding board for the methods discussed
in this paper.

\bibliography{pisn-line}

\end{document}
