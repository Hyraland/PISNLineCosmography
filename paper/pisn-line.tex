\documentclass[modern]{aastex62}

\usepackage{acro}

% Commands
\newcommand{\fixme}[1]{\textcolor{red}{#1}}

% Shorthand
\newcommand{\mdet}{m^\mathrm{detector}}
\newcommand{\monedet}{m_1^\mathrm{detector}}
\newcommand{\msource}{m^\mathrm{source}}

% Quantities
\newcommand{\fiveYearUncert}{5 \, \kmsMpc}
\newcommand{\MSourceMax}{40 \, \MSun}
\newcommand{\None}{282}
\newcommand{\Nfive}{1463}
\newcommand{\oneYearUncert}{10 \, \kmsMpc}

% units
\newcommand{\Gpc}{\mathrm{Gpc}}
\newcommand{\kmsMpc}{\mathrm{km} \, \mathrm{s}^{-1} \, \mathrm{Mpc}^{-1}}
\newcommand{\MSun}{M_\odot}

% Acronym defn's
\DeclareAcronym{BH}{
  short = BH,
  long = {black hole}
}
\DeclareAcronym{BBH}{
  short = BBH,
  long = {binary black hole}
}
\DeclareAcronym{BNS}{
  short = BNS,
  long = {binary neutron star}
}
\DeclareAcronym{GW}{
  short = GW,
  long = {gravitational wave}
}
\DeclareAcronym{PISN}{
  short = PISN,
  long = {pair instability supernova}
}

\begin{document}

\title{The Pair Instability Supernova Mass Absorption Line: Enabling a Pure
Gravitational Wave Measurement of the Hubble Constant}

\author[0000-0003-1540-8562]{Will M. Farr}
\affiliation{Department of Physics and Astronomy, Stony Brook University, Stony Brook NY 11794, USA}
\affiliation{Center for Computational Astronomy, Flatiron Institute, 162 5th Ave., New York NY 10010, USA}
\email{will.farr@stonybrook.edu}

\author[0000-0002-1980-5293]{Maya Fishbach}
\affiliation{Department of Astronomy and Astrophysics, University of Chicago, Chicago IL 60637, USA}
\email{mfishbach@uchicago.edu}

\author[0000-0002-0175-5064]{Daniel E. Holz}
\affiliation{Enrico Fermi Institute, Department of Physics, Department of Astronomy and Astrophysics,\\and Kavli Institute for Cosmological Physics, University of Chicago, Chicago IL 60637, USA}
\email{holz@uchicago.edu}

\begin{abstract}
%
  Gravitational interactions are scale-free.  Because of this freedom, there is
  a perfect degeneracy between the mass scale and redshift of a gravitational
  wave source; some non-gravitational process is necessary to provide redshift
  information. \Ac{GW} observations do provide direct information about the
  luminosity distance of a \ac{GW} source.  Here we suggest that the \ac{PISN}
  process, thought to impose a sharp upper-limit on the source-frame mass of
  \ac{BH} stellar remnants, provides an ``absorption line'' that permits a
  statistical measurement of the redshift-luminosity-distance relation using a
  population of \ac{BBH} inspirals.  Assuming that the intrinsic scatter in the
  mass upper-limit of the \ac{BH} mass distribution imposed by the \ac{PISN}
  process is $\lesssim 1\,\MSun$, we simulate one and five years of Advanced
  LIGO \ac{BBH} detections with realistic uncertainties is mass and distance
  measurements and show that analysis of these populations permits determination
  of the Hubble constant to $\pm \oneYearUncert$ after one and $\pm
  \fiveYearUncert$ after five years of observation.  These uncertainties are
  competitive with, if not better, than obtainable through any other known
  \ac{GW} observational technique with this data set, including observations of
  \ac{BNS} mergers \emph{with} electromagnetic counterparts like GW170817.
%
\end{abstract}

\section{Introduction}

Some collected references (for use later):

\begin{itemize}
  \item \ac{PISN} \cite{Heger2002,Belczynski2016,Woosley2017,Spera2017}.
  \item GW170817: \cite{GW170817,GW170817-H0}
  \item Accuracy of $H_0$ measurements: \cite{GW170817-H0,Chen2017}
\end{itemize}

\section{Method}
\label{sec:method}

\Ac{GW} detectors are directly sensitive to the mass and redshift combination
%
\begin{equation}
  \label{eq:Mdet-definition}
  \mdet \equiv \msource \left( 1 + z \right),
\end{equation}
%
where $\mdet$ is the observed mass, $\msource$ is the source-frame mass, and $z$
is the redshift of the source.  Comparing parameterised \ac{GW} waveforms
\citep[e.g.][]{Taracchini2014,Bohe2017,Smith2016,Kahn2016,Chatziioannou2017} to
the data observed by \ac{GW} detectors it is possible to produce a posterior
distribution over \ac{BBH} parameters that describes our uncertainty about the
source properties \citep{Veitch2015,GW150914-PE}; in particular, one can obtain
a marginalized two-dimensional distribution describing our knowledge of
$\monedet$ and $d_L$, ignoring all other parameters (here we follow the standard
convention of denoting the mass of the more massive component of the binary by
$m_1$).  Rather than simulate this process for synthetic events, we here
approximate the posterior that would be obtained using a method similar to
\citet{Fishbach2018}, but tuned to produce uncertainties in these two parameters
that are comparable to GW170814 \citep{GW170814} at comparable signal-to-noise
ratios.

\fixme{Boring details about mass ratio distribution, redshift evolution, etc,
etc.}

We generate catalogs of events that correspond to all detected \ac{BBH} events
from one year and five years of three-detector observing time at design
sensitivity \citep{AdvancedLIGO,AdvancedVIRGO}.  The first year's observations
of the five-year catalog are identical to the one-year catalog. The catalogs
contain \None{} and \Nfive{} events.  Figure \ref{fig:true-det-masses-dLs} shows
the true (detector frame) masses and luminosity distances for the detected
events; the events range from $\sim 5 \, \MSun$ to $80 \, \MSun$ in mass
(detector frame) and reach to $d_L \sim 6 \, \Gpc$.  Our approximation to the
posterior produces the set of inferred masses and distances shown in Figure
\ref{fig:obs-det-masses-dLs}.

\begin{figure}
  \plotone{plots/m1-dL-true}
%
  \caption{\label{fig:true-det-masses-dLs} The true (detector frame) masses and
  luminosity distances for the one year (dark) and five year (light) catalogs
  described in Section \ref{sec:method}.  The effect of redshifting on the mass
  can clearly be seen; the maximum source-frame mass is $\MSourceMax$.  The
  solid black line gives the maximum detector-frame mass as a function of
  luminosity distance for the cosmology used to generate the catalog
  \citep{Planck2016}; the dashed lines give the same relation but for a 10\%
  change in $H_0$.}
%
\end{figure}

\begin{figure}
  \plotone{plots/m1-dL-obs}
%
  \caption{\label{fig:obs-det-masses-dLs} The inferred (detector frame) masses
  and luminosity distances for the one-year catalog (dark) and five-year catalog
  (light).  The points are the posterior mean and bars give the posterior
  standard deviation with a flat prior on $\monedet$ and $d_L$.  The solid black
  line gives the maximum detector-frame mass as a function of luminosity
  distance for the cosmology used to generate the catalog.}
%
\end{figure}

\section{Conclusions}

\acknowledgments

We thank Stephen Feeney for providing a sounding board for the methods discussed
in this paper.

\bibliography{pisn-line}

\end{document}
